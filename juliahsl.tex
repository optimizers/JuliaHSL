\documentclass[gdweb]{geradwp}

\PassOptionsToPackage{hyphens}{url}
\usepackage[ruled,linesnumbered]{algorithm2e}
\usepackage{hyperref}

\GDcoverpagewhitespace{6.8cm}
\graphicspath{{Figures/}}
\hypersetup{colorlinks,%
citecolor={blue},
urlcolor={blue},
linkcolor={blue},
breaklinks={true}
}

\makeatletter
\ifthenelse{\isundefined{\ALG@name}}{}%
{%
\renewcommand{\ALG@name}{\sffamily\footnotesize Algorithm}
}
\makeatother
\ifthenelse{\isundefined{\SetAlCapNameFnt}}{}%
{%
\SetAlCapNameFnt{\footnotesize}
\SetAlCapFnt{\sffamily\footnotesize}
}

%% %%%%%%%%%%%%%%%%%%%%%%%%%%%%%%%%%%%%%%%%%%%%%%%%%
%% %%%%%% Début - commandes de l'auteur %%%%%%%%%%%%
%% %%%%%%%% Start of author commands %%%%%%%%%%%%%%%
%% %%%%%%%%%%%%%%%%%%%%%%%%%%%%%%%%%%%%%%%%%%%%%%%%%

% Package verbatim
\usepackage{verbatim}

% Package natbib
\usepackage{natbib}

% Package jlcode
\usepackage[charsperline=92,theme=default-plain]{jlcode}

% Commande LaTeX pour rajouter les liens DOI
\newcommand*{\doi}[1]{DOI: \href{http://dx.doi.org/\detokenize{#1}}{\detokenize{#1}}}
\newcommand*{\doilink}[2]{\href{http://dx.doi.org/\detokenize{#1}}{#2}}

% Commande LaTeX pour faciliter le remplacement des liens Markdown [...](...)
\newcommand{\reversedhref}[2]{\href{#2}{#1}}

\newcommand{\HSLjll}{HSL\_jll.jl}
\newcommand{\Ipoptjll}{Ipopt\_jll.jl}
\newcommand{\LBTjll}{libblastrampoline\_jll.jl}
\newcommand{\OpenBLASjll}{OpenBLAS32\_jll.jl}

%% %%%%%%%%%%%%%%%%%%%%%%%%%%%%%%%%%%%%%%%%%%%%%%%%%
%% %%%%%%% Fin - commandes de l'auteur %%%%%%%%%%%%%
%% %%%%%%%%% End of author commands %%%%%%%%%%%%%%%%
%% %%%%%%%%%%%%%%%%%%%%%%%%%%%%%%%%%%%%%%%%%%%%%%%%%

%% %%%%%%%%%%%%%%%%%%%%%%%%%%%%%%%%%%%%%%%%%%%%%%%%%
%% %%%%%%%%%% Métadonnées du cahier  %%%%%%%%%%%%%%%
%% %%%%%%%%%% Working paper metadata %%%%%%%%%%%%%%%
%% %%%%%%%%%%%%%%%%%%%%%%%%%%%%%%%%%%%%%%%%%%%%%%%%%

\GDtitre{\texttt{JuliaHSL.jl}: the ultimate collection for large scale scientific computation}
\GDmois{Mai}{May}
\GDannee{2023}
\GDnumero{XX}
\GDauteursCourts{A. Montoison, J. Fowkes, A. Lister, D. Orban}
\GDauteursCopyright{Montoison, Fowkes, Lister, Orban}
\GDpostpubcitation{J. Fowkes, A. Lister, A. Montoison, D. Orban (2023). ``Un exemple de citation'', \emph{Journal of Journals}, vol. X issue Y, p. n-m}{https://www.gerad.ca/fr}
\GDsupplementname{Internet Appendix}

\begin{document}

\GDpageCouverture

\begin{GDpagetitre}

\begin{GDauthlist}
    \GDauthitem{Jaroslav Fowkes \ref{affil:2}}
    \GDauthitem{Andrew Lister \ref{affil:2}}
    \GDauthitem{Alexis Montoison \ref{affil:1}}
    \GDauthitem{Dominique Orban \ref{affil:1}}
\end{GDauthlist}

\begin{GDaffillist}
    \GDaffilitem{affil:1}{GERAD and Department of Mathematics and Industrial Engineering, Polytechnique Montréal, Montr\'eal (Qc), Canada, H3T 2A7}
    \GDaffilitem{affil:2}{Scientific Computing Department, Rutherford Appleton Laboratory, Chilton, England}
\end{GDaffillist}

\begin{GDemaillist}
    \GDemailitem{jaroslav.fowkes@stfc.ac.uk}
    \GDemailitem{andrew.lister@stfc.ac.uk}
    \GDemailitem{alexis.montoison@polymtl.ca}
    \GDemailitem{dominique.orban@gerad.ca}
\end{GDemaillist}

\end{GDpagetitre}

%% %%%%%%%%%%%%%%%%%%%%%%%%%%%%%%%%%%%%%%%%%%%%%%%%%%%%%%%
%% %%%%%%%%% Résumés, mots-clés, remerciements %%%%%%%%%%%
%% %%%%%%% Abstract, keywords, acknowledgements %%%%%%%%%%
%% %%%%%%%%%%%%%%%%%%%%%%%%%%%%%%%%%%%%%%%%%%%%%%%%%%%%%%%

\GDabstracts

\begin{GDabstract}{Abstract}
This paper presents \texttt{JuliaHSL}.

\paragraph{Keywords: }
Julia, numerical linear algebra
\end{GDabstract}

\begin{GDabstract}{R\'esum\'e}
Cet article pr\'esente \texttt{JuliaHSL}.

\paragraph{Mots cl\'es\,: }
Julia, alg\`ebre lin\'eaire num\'erique
\end{GDabstract}

\begin{GDacknowledgements}
Alexis Montoison is supported by an FRQNT grant and an excellence scholarship of the IVADO institute, and Dominique Orban is partially supported by an NSERC Discovery Grant.

This work began while Alexis Montoison was visiting Nick Gould, Jaroslav Fowkes and the \href{https://www.numerical.rl.ac.uk/}{Computational Mathematics Group} at the \href{https://www.ukri.org/about-us/stfc/locations/rutherford-appleton-laboratory/}{STFC Rutherford Appleton Laboratory} in December 2022.
He expresses its appreciations to the HSL team for their invitation and their wonderful hospitality.

The authors would like to thank Stefan Vigerske for the \href{https://github.com/KarypisLab/METIS}{METIS} adapter, Geoffroy Leconte, Paul Raynaud, and Oscar Dowson for testing early versions of \HSLjll, Mosè Giordano, and Elliot Saba for their work on \href{https://github.com/JuliaPackaging/BinaryBuilder.jl}{BinaryBuilder.jl}, \href{https://github.com/JuliaLinearAlgebra/libblastrampoline}{libblastrampoline} and \href{https://github.com/JuliaPackaging/Yggdrasil}{Yggdrasil}.
\end{GDacknowledgements}

%% %%%%%%%%%%%%%%%%%%%%%%%%%%%%%%%%%%%%%%%%%%%%%%%%%
%% %%%%%%%%%%%%%%%% Article %%%%%%%%%%%%%%%%%%%%%%%%
%% %%%%%%%%%%%%%%%%%%%%%%%%%%%%%%%%%%%%%%%%%%%%%%%%%

\GDarticlestart

\section{Introduction}

JuliaHSL is the ultimate sparse linear algebra collection for large-scale scientific computing.
It contains more than 155 HSL packages and aims to facilitate the use of HSL in \href{https://julialang.org/}{Julia}.
JuliaHSL also integrates easily into Fortran and C projects such as \href{https://github.com/ralna/GALAHAD}{GALAHAD}.

JuliaHSL focuses on ease of use for users on all platforms by using a Meson build system.
Meson enables the distribution of prebuilt binaries for the package using BinaryBuilder.jl.
Additionally, JuliaHSL supports either METIS 5 or the older METIS 4.

This new package provides the source code of the included HSL packages as well as a Julia package named \HSLjll.
\HSLjll~is a pre-built version of JuliaHSL to be readily used in the Julia ecosystem.
Once \HSLjll~is installed, the shared library that contains the C and Fortran routines of the HSL packages can be called in Julia and the HSL wrappers provided in the Julia interface HSL.jl are functional.
\HSLjll~also provides an easy way to use the HSL linear solvers MA27, MA57, MA77, MA86 and MA97 within the IPOPT.jl interface to the IPOPT nonlinear optimization solver.

Two versions of \HSLjll~are available. One precompiled with OpenBLAS that requires at least Julia 1.6 and the other version precompiled with libblastrampoline (LBT) that requires at least Julia 1.9.
LBT allows one to dynamically switch the BLAS and LAPACK backends between e.g.\ OpenBLAS, BLIS, Intel MKL or Apple Accelerate.

For information on how to use each function available through this package directly please see the relevant documentation on \href{https://www.hsl.rl.ac.uk/catalogue/}{the HSL website}.
Where C interfaces are available, the C documentation should be used.

\section{Building JuliaHSL with Meson}

\subsection{Meson}

To download and install \href{https://mesonbuild.com}{Meson}, we refer the user to this \href{https://mesonbuild.com/SimpleStart.html}{guide}.
By default Meson uses the \href{https://ninja-build.org/}{Ninja} build system.
After configuration, it is equivalent to build the package or install it directly with Ninja.

\subsection{Configuration}

\begin{jllisting}
meson setup builddir [options...]
CC=icc FC=ifort meson setup builddir [options...]
\end{jllisting}
creates the build directory \jlinl{builddir} and populates it in preparation for a build.
Non-default compilers can be selected by setting the \jlinl{CC} and \jlinl{FC} shell variables.
See this \href{https://mesonbuild.com/Reference-tables.html}{table} for supported compilers.
\begin{table}[ht]
  \label{tab:options}
  \centering
  \begin{tabular}{|c|c|c|}
    \hline
    option              & default value & description                      \\ \hline
    -Dmodules           & true          & install the Fortran module files \\ \hline
    -Dlibmetis\_version & 5             & version of the METIS library     \\ \hline
    -Dlibblas           & blas          & name of a BLAS library           \\ \hline
    -Dliblapack         & lapack        & name of a LAPACK library         \\ \hline
    -Dlibmetis          & metis         & name of a METIS library          \\ \hline
    -Dlibmpi            & mpifort       & name of a MPI library            \\ \hline
    -Dlibblas\_path     & [~]           & location of the BLAS library     \\ \hline
    -Dliblapack\_path   & [~]           & location of the LAPACK library   \\ \hline
    -Dlibmetis\_path    & [~]           & location of the METIS library    \\ \hline
    -Dlibmpi\_path      & [~]           & location of the MPI library      \\ \hline
    -Dlibmetis\_include & [~]           & location of the METIS headers    \\ \hline
    -Dlibmpi\_include   & [~]           & location of the MPI headers      \\ \hline
  \end{tabular}
  \caption{Options supported by JuliaHSL}
\end{table}

\subsection{Compilation}

\begin{jllisting}
# Meson
meson compile -C builddir

# Ninja
ninja -C builddir
\end{jllisting}
compiles JuliaHSL in the \jlinl{builddir} folder.

\subsection{Installation}

\begin{jllisting}
# Meson
meson install -C builddir

# Ninja
ninja -C builddir install
\end{jllisting}
installs the library, headers and Fortran modules in \jlinl{C:/} on Windows, and \jlinl{usr/local} otherwise.
This can be changed by passing the command line argument \jlinl{--prefix} to Meson during configure time.
\begin{jllisting}
meson setup builddir --prefix=~/juliahsl
\end{jllisting}

\subsection{Cross-compilation}

JuliaHSL can be easily cross-compiled with the Julia package \href{https://github.com/JuliaPackaging/BinaryBuilder.jl}{BinaryBuilder.jl}.
We provide the script \jlinl{build\_tarballs.jl} to easily regenerate an \HSLjll~with different options or dependencies, such as METIS 4 instead of METIS 5.
It can also be used to generate a static library \jlinl{libhsl.a} for an application outside of Julia.
Commands to cross-compile JuliaHSL are available in the file \jlinl{build\_tarballs.jl}.

\section{Use JuliaHSL in C and Fortran projects}

Once JuliaHSL is compiled, you can link it to your C and Fortran projects with
\begin{jllisting}
# C
-I${prefix}/include -L${prefix} -lhsl

# Fortran
-I${prefix}/modules -I${prefix}/include -L${prefix} -lhsl
\end{jllisting}
By default Meson generates a shared library but if you prefer to do static linking, the Meson's option \jlinl{--default-library=static} can be used to generate a static library \jlinl{libhsl.a}.

\section{Use \HSLjll~in the Julia ecosystem}

\subsection{Installation of the \HSLjll~package}

To install the package \HSLjll, you only need the following commands in the Julia REPL:
\begin{jllisting}
julia> ]
pkg> dev path_to_hsl_jll
\end{jllisting}
We recommend using the version precompiled with \LBTjll~if you use Julia 1.9 or future stable releases.
The version precompiled with \OpenBLASjll~can be used in the long-term support (LTS) release Julia 1.6.
Due to security policy of the macOS operating system, the Mac users may need to remove the quarantine before extracting.
\begin{jllisting}
xattr -d com.apple.quarantine HSL_jll.jl-XXXX.YY.ZZ.zip
xattr -d com.apple.quarantine HSL_jll.jl-XXXX.YY.ZZ.tar.gz
\end{jllisting}

\subsection{Load an LP64 BLAS and LAPACK backend}

By default Julia ships with only an ILP64 version of OpenBLAS as BLAS and LAPACK backend.
ILP64 libraries use the 64-bit integer type (necessary for indexing large arrays, with more than $2^{31}-1$ elements), whereas the LP64 libraries index arrays with the 32-bit integer type.
JuliaHSL can only be linked with LP64 versions of BLAS and LAPACK as HSL uses 32-bit integer index arrays.
Thus, one version of \HSLjll~is precompiled with \OpenBLASjll~and automatically loads an LP64 version of OpenBLAS with a \jlinl{using HSL_jll}.
For the version precompiled with \LBTjll~, the user needs to manually load one LP64 BLAS and LAPACK backend except if \HSLjll~is used with \href{https://github.com/JuliaSmoothOptimizers/HSL.jl}{HSL.jl} or \href{https://github.com/jump-dev/Ipopt.jl}{Ipopt.jl}.
\begin{jllisting}
# Load the LP64 version of OpenBLAS
import LinearAlgebra, OpenBLAS32_jll
LinearAlgebra.BLAS.lbt_forward(OpenBLAS32_jll.libopenblas_path)

# Load the LP64 version of Intel MKL
import LinearAlgebra, MKL_jll
LinearAlgebra.BLAS.lbt_forward(MKL_jll.libmkl_rt_path)
\end{jllisting}
We can verify what backends are loaded with \jlinl{LinearAlgebra.BLAS.lbt_get_config()}.
The LP64 and ILP64 versions of Intel MKL can also be loaded with the Julia package \href{https://github.com/JuliaLinearAlgebra/MKL.jl}{MKL.jl} and a \jlinl{using MKL}.
In the future, an option to use BLIS and Apple Accelerate LP64 versions of BLAS and LAPACK will be available with the \href{https://github.com/JuliaLinearAlgebra/BLISBLAS.jl}{BLISBLAS.jl} and \href{https://github.com/JuliaMath/AppleAccelerate.jl}{AppleAccelerate.jl} packages.
The user can also use their own LP64 version of BLAS and LAPACK.

\subsection{How to use \HSLjll~in Julia packages}

A version 1.0 of \HSLjll~based on this dummy \href{https://github.com/ralna/JuliaHSL}{JuliaHSL} was precompiled with \href{https://github.com/JuliaPackaging/Yggdrasil}{Yggdrasil}.
Therefore \HSLjll~is a registered Julia package and can be added as a dependency of any Julia package.
The dummy version only has one C function \jlinl{JULIAHSL\_isfunctional} that returns the boolean \jlinl{false}.
The real version also exports this C function and returns the boolean \jlinl{true}.
\begin{jllisting}
using HSL_jll

function JULIAHSL_isfunctional()
    @ccall libhsl.JULIAHSL_isfunctional()::Bool
end

bool = JULIAHSL_isfunctional()
\end{jllisting}
This function makes it possible to determine if we have a dummy version or not and consequently to call the C and Fortran routines of the genuine version.

The package \href{https://github.com/JuliaSmoothOptimizers/HSL.jl}{HSL.jl} provides wrappers for all HSL packages with a C interface or those written in Fortran 77.
Wrappers are Julia functions that call C and Fortran routines with the macro \jlinl{@ccall}.
Thus, the combination of HSL.jl and \HSLjll~allows one to abstract away from the low-level C and Fortran  languages and use the majority of HSL packages as if they were packages developed in Julia.
% High-level Julia functions that exploit the multiple dispatch feature are underway.
Note that HSL.jl also loads the LP64 version of OpenBLAS automatically for the user if needed.

For the Julia interface \href{https://github.com/jump-dev/Ipopt.jl}{Ipopt.jl}, \HSLjll~must be used directly because it interacts with the \Ipoptjll~package under the hood.

\begin{jllisting}
using JuMP
import Ipopt, HSL_jll

# An optimization problem
model = Model(Ipopt.Optimizer)
@variable(model, x)
@objective(model, Min, (x - 2)^2)

# Load the HSL solvers
set_attribute(model, "hsllib", HSL_jll.libhsl_path)

# Use the linear solver MA57
set_attribute(model, "linear_solver", "ma57")
optimize!(model)

# Use the linear solver MA97
set_attribute(model, "linear_solver", "ma97")
optimize!(model)
\end{jllisting}

The available HSL linear solvers are \jlinl{"ma27"}, \jlinl{"ma57"}, \jlinl{"ma77"}, \jlinl{"ma86"} and \jlinl{"ma97"}.
If no LP64 BLAS and LAPACK backend is loaded, Ipopt.jl loads \OpenBLASjll~automatically just like HSL.jl.

\section{Bug reports and discussions}

If you think you have found a bug, feel free to open an \href{https://github.com/ralna/JuliaHSL/issues}{issue}.
Useful suggestions and requests can also be opened as issues.
If you would like to ask a question not suitable for a bug report, feel free to start a \href{https://github.com/ralna/JuliaHSL/discussions}{discussion}.

% \bibliographystyle{abbrvnat}
% \bibliography{abbrv,juliahsl}

\end{document}

\endinput
